\documentclass[a4paper,10pt]{article}

\usepackage{../../latex/mystyle}
\usepackage[top=3cm, bottom=3cm, left=3cm, right=3cm]{geometry}

\begin{document}

%\section*{Extensions choisies}

\begin{center}
\begin{tabu} to \textwidth {lX[c]r}
    Michel de Broux & \large{\textbf{Choix des extensions}} & Charles Momin \\
    Simon Lardinois & LSINF1225 & Valentin Rombouts \\
    Victor Lecomte & Groupe V & Harold Somers \\
    \hline
\end{tabu}
\end{center}

\vspace{0.5cm}

\begin{enumerate}
    \item \textbf{Recherche avancée:}
    
    \item \textbf{Produits composés:}
    
    Appelée «restaurant» dans le cahier des charges, cette extension consiste à gérer des consommations composées de plusieurs ingrédients, comme des cocktails. Par exemple, un mazout peut utiliser 17\,cl de bière et 8\,cl de Coca-Cola. Cette extension peut être facilement étendue à de la petite restauration.
\end{enumerate}

\end{document}
