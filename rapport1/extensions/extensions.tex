\documentclass[a4paper,10pt]{article}

\usepackage{../../latex/mystyle}
\usepackage[top=3cm, bottom=3cm, left=3cm, right=3cm]{geometry}

\begin{document}

\header{Choix des extensions}

\begin{enumerate}
    \item \textbf{Produits composés:}
    
    Appelée «restaurant» dans le cahier des charges, cette extension consiste à gérer des consommations composées de plusieurs ingrédients, comme des cocktails. Par exemple, un mazout peut utiliser 17\,cl de bière et 8\,cl de Coca-Cola. Cette extension pourrait être facilement appliquée à de la petite restauration.
    
    \item \textbf{Graphiques:}
    
    Toutes sortes de visualisations de données, principalement destinées au manager. Par exemple, des graphes sur l'évolution des recettes, l'efficacité comparée des serveurs par boisson servie au long de la soirée, sur l'approvisionnement du stock, et bien d'autres.
\end{enumerate}

\end{document}
