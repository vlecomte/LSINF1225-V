\documentclass[a4paper,10pt]{article}

\usepackage{../../latex/mystyle}
\usepackage[top=1cm, bottom=2cm, left=2cm, right=2cm]{geometry}

\begin{document}

\header{Faits élémentaires}

\begin{center}
\begin{tabu}{lll}
    Membre 1 & Relation & Membre 2 \\
    \toprule  
    \entity{Utilisateur}{login}{tiffany} & est de grade & \val{"client"} \\
    \entity{Utilisateur}{login}{tiffany} & possède l'e-mail & \val{"tiffany@hotmail.be"} \\
    \entity{Utilisateur}{login}{tiffany} & a le mot de passe & \val{"karateswag22"} \\
    \entity{Utilisateur}{login}{tiffany} & utilise la langue & \val{"français"} \\
    \entity{Utilisateur}{login}{tiffany} & commande & \entity{Commande}{numéro}{42} \\
    \entity{Utilisateur}{login}{tiffany} & commande & \entity{Commande}{numéro}{43} \\
    \entity{Utilisateur}{login}{jeff} & est de grade & \val{"serveur"} \\
    \entity{Utilisateur}{login}{jeff} & possède l'e-mail & \val{"jeff@hotmail.be"} \\
    \entity{Utilisateur}{login}{jeff} & a le mot de passe & \val{"ilovetiffany"} \\
    \entity{Utilisateur}{login}{jeff} & utilise la langue & \val{"anglais"} \\
    \entity{Commande}{numéro}{42} & est à la table & \val{1} \\
    \entity{Commande}{numéro}{42} & a été payée à & \val{21/12/2012 11:53} \\
    \entity{Commande}{numéro}{43} & est à la table & \val{3} \\
    \textcolor{lightgray}{\entity{Commande}{numéro}{43}} & \textcolor{lightgray}{a été payée à} & \textcolor{lightgray}{(pas encore)\footnotemark[1]} \\
    \entitytwo{Détail}{commande}{42}{heure}{11:45} & est un(e) & \entity{Consommation}{nom}{Maes 25cl} \\
    \entitytwo{Détail}{commande}{42}{heure}{11:45} & a été commandé? & \val{oui} \\
    \entitytwo{Détail}{commande}{42}{heure}{11:45} & a été servie par & \entity{Utilisateur}{login}{jeff} \\
    \entitytwo{Détail}{commande}{43}{heure}{18:01} & est un(e) & \entity{Consommation}{nom}{Gin tonic} \\
    \entitytwo{Détail}{commande}{43}{heure}{18:01} & a été commandé? & \val{non} \\
    \textcolor{lightgray}{\entitytwo{Détail}{commande}{43}{heure}{18:01}} & \textcolor{lightgray}{a été servie par} & \textcolor{lightgray}{(pas encore)\footnotemark[1]} \\
    \entity{Consommation}{nom}{Maes 25cl} & coûte & \val{1€80} \\
    \entity{Consommation}{nom}{Maes 25cl} & est de type & \entity{Type}{nom}{bière} \\
    \entity{Consommation}{nom}{Maes 25cl} & est décrit(e) par & \entity{Description}{titre}{Maes} \\
    \entity{Consommation}{nom}{Maes 25cl} & utilise \val{1} de\footnotemark[2] & \entity{Ingrédient}{nom}{Maes 25cl} \\
    \entity{Consommation}{nom}{Gin tonic} & coûte & \val{6€00} \\
    \entity{Consommation}{nom}{Gin tonic} & est de type & \entity{Type}{nom}{cocktail} \\
    \entity{Consommation}{nom}{Gin tonic} & est décrit(e) par & \entity{Description}{titre}{Gin tonic} \\
    \entity{Consommation}{nom}{Gin tonic} & utilise \val{1} de\footnotemark[2] & \entity{Ingrédient}{nom}{Tonic} \\
    \entity{Consommation}{nom}{Gin tonic} & utilise \val{0.05} de\footnotemark[2] & \entity{Ingrédient}{nom}{Gin} \\
    \entity{Type}{nom}{bière} & est illustré par & \val{"biere.png"} \\
    \entity{Type}{nom}{cocktail} & est illustré par & \val{"cocktail.png"} \\
    \entity{Description}{nom}{Maes} & est illustrée par & \val{"maes.jpg"} \\
    \entity{Description}{nom}{Maes} & est décrite par & \val{"maes.txt"} \\
    \entity{Description}{nom}{Gin tonic} & est illustrée par & \val{"Gin tonic.jpg"} \\
    \entity{Description}{nom}{Gin tonic} & est décrite par & \val{"Gin tonic.txt"} \\
    \entity{Ingrédient}{nom}{Maes 25cl} & a les unités & \val{"bouteilles"} \\
    \entity{Ingrédient}{nom}{Maes 25cl} & est en quantité & \val{33} \\
    \textcolor{lightgray}{\entity{Ingrédient}{nom}{Maes 25cl}} & \textcolor{lightgray}{est en quantité max.} & \textcolor{lightgray}{(pas de max)\footnotemark[1]} \\
    \entity{Ingrédient}{nom}{Maes 25cl} & est en quantité seuil & \val{20} \\
    \entity{Ingrédient}{nom}{Tonic} & a les unités & \val{"bouteilles"} \\
    \entity{Ingrédient}{nom}{Tonic} & est en quantité & \val{6} \\
    \entity{Ingrédient}{nom}{Tonic} & est en quantité max. & \val{25} \\
    \entity{Ingrédient}{nom}{Tonic} & est en quantité seuil & \val{10} \\
    \entity{Ingrédient}{nom}{Gin} & a les unités & \val{"litres"} \\
    \entity{Ingrédient}{nom}{Gin} & est en quantité & \val{1.5} \\
    \entity{Ingrédient}{nom}{Gin} & est en quantité max. & \val{7} \\
    \entity{Ingrédient}{nom}{Gin} & est en quantité seuil & \val{1} \\
\end{tabu}
\end{center}

\footnotetext[1]{Les lignes en gris désignent des faits élémentaires absents. (On ne peut pas utiliser de négations.)}
\footnotetext[2]{Cette relation est ternaire, le troisième membre étant la quantité utilisée.}

\end{document}
