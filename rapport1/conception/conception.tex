\documentclass[a4paper,10pt]{article}

\usepackage{../../latex/mystyle}
\usepackage[top=3cm, bottom=3cm, left=3cm, right=3cm]{geometry}

\begin{document}

\header{Choix de conception}

\section{Différences par rapport à l'énoncé}

La différence principale de notre projet par rapport à l'énoncé est que le client peut lui-même commander des boissons et demander l'addition. Il n'a pas besoin de passer par l'intérmédiaire du serveur, bien que ce soit possible: dans ce cas les commandes ne seront pas associées à un utilisateur, mais seulement à la table.

\section{Description des entités}

\begin{tabu}{X}
\textbf{Utilisateur} \\
\toprule
Pour notre projet, nous avons décidé qu'il serait possible au client de commander lui-même des boissons. Dès lors, les utilisateurs de l'application peuvent être de trois types: client, serveur et manager. Ces trois rôles auront chacun leurs privilèges et leurs écrans. Afin d'utiliser l'application, un utilisateur doit s'identifier via un écran de login.

\textsl{Exemples: Tiffany, une cliente, et Jeff, un serveur.}

Les utilisateurs sont identifiés de manière unique par leur login. \\\\
\end{tabu}

\begin{tabu}{X}
\textbf{Commande} \\
\toprule

Dans notre conception, une commande représente l'ensemble des boissons consommées par un utilisateur entre deux paiements. Quand l'utilisateur commande une boisson, elle est ajoutée à la commande courante, et une fois que l'utilisateur paie, cette commande est fermée et les boissons suivantes, le cas échéant, seront ajouté à une nouvelle commande.

\textsl{Exemples: la commande \no42 par Tiffany pour sa pose de midi, elle a bu une Maes et a déjà payé; et la commande \no43 après son travail, elle a ajouté un Gin Tonic et n'a pas encore payé.}

Les commandes sont identifiées de manière unique par leur numéro de commande. \\\\
\end{tabu}

\begin{tabu}{X}
\textbf{Détail} \\
\toprule

Un détail représente une boisson ajoutée à une commande. Quand plusieurs boissons identiques sont commandées, nous créons un détail pour chacune d'entre elles. Nous considérons qu'il n'y a là pas de réelle redondance: en effet, les détails peuvent avoir été ajoutés ou servis à des moments différents, et certains d'entre eux peuvent avoir déjà été servis tandis que d'autres viennent seulement d'être ajoutés. De plus, la gestion des détails sous cette forme est significativement plus simple.

\textsl{Exemples: le détail de la commande \no42 ajouté à 11h45, une Maes 25cl, servie par Jeff à 11h47; et le détail de la commande \no43 ajouté à 18h01, un gin tonic, pas encore servi.}

Les détails sont identifiés de manière unique par la commande dont ils font partie et l'heure d'ajout du détail, pris ensemble. \\\\
\end{tabu}

\begin{tabu}{X}
\textbf{Consommation} \\
\toprule

Une consommation représente une boisson sur la carte. Si plusieurs boissons identiques sont commandées, elles pointent toutes vers la même consommation. Les consommations sont liées notamment à des types de consommation, des descriptions et des ingrédients pour les spécifier.
    
\textsl{Exemples: la consommation Maes 25cl, une bière qui coûte 1.80€; et la consommation Gin Tonic, un cocktail à 6.00€ composé de Gin et de Tonic.}

Les consommations sont identifiées de manière unique par leur nom de consommation. \\\\
\end{tabu}

\begin{tabu}{X}
\textbf{Type} \\
\toprule

Un type représente une catégorie de consommation. Le type peut-être affiché sur la carte sous forme d'une icône.

\textsl{Exemples: le type bière, le type cocktail.}

Les types sont identifiés de manière unique par leur nom de type. \\\\
\end{tabu}

\begin{tabu}{X}
\textbf{Description} \\
\toprule

Une description représente le texte descriptif d'un produit et son image. Des consommations qui ne diffèrent que par la taille mais qui sont le même produit vont partager la même description.

\textsl{Exemples: la description Maes, avec comme image une bouteille de Maes, qui décrit aussi bien un Maes 25cl qu'une Maes 30cl; la description Gin Tonic, qui ne décrit que le Gin Tonic.}

Les descriptions sont identifiées de manière unique par leur nom de description. \\\\
\end{tabu}

\begin{tabu}{X}
\textbf{Ingrédient} \\
\toprule

Un ingrédient représente le stock d'une matière première, par bouteille ou en vrac. Si une consommation utilise plusieurs ingrédients différents, il sera lié à ces ingrédients avec des relations indiquant aussi la quantité consommée. L'ingrédient possède des informations sur le stock actuel, le seuil et le maximum.

\textsl{Exemples: l'ingrédient Maes 25cl, utilisé une bouteille à la fois; l'ingrédient Gin, utilisé par petites quantités dans le Gin Tonic.}

Les ingrédients sont identifiés de manière unique par leur nom d'ingrédient. \\\\
\end{tabu}

\section{Précisions sur les relations}

\begin{tabu}{X}
\textbf{Liées à Utilisateur}\\
\toprule

\begin{itemize}
    \item possède le login: l'utilisateur étant identifié par son login, cette relation est obligatoire pour l'utilisateur, l'utilisateur a un seul login, et un login appartient à un seul utilisateur.
    \item possède l'e-mail: l'e-mail étant utilisé notamment pour récupérer un mot de passe oublié, cette relation est obligatoire pour l'utilisateur, l'utilisateur a un seul e-mail, et un e-mail appartient à un seul utilisateur.
    \item a le mot de passe: pour des raisons de sécurité, le mot de passe est obligatoire pour l'utilisateur, et un utilisateur ne peut avoir qu'un mot de passe; un mot de passe peut par contre être utilisé par plusieurs utilisateurs.
    \item est de grade: un utilisateur doit avoir un des trois grades \val{"client"}, \val{"serveur"} et \val{"admin"}, donc le grade est obligatoire pour l'utilisateur, et un utilisateur ne peut avoir qu'un grade; mais plusieurs utilisateurs peuvent avoir le même grade.
    \item utilise la langue: la langue indiquée sera utilisée pour afficher l'application, par conséquent la langue est obligatoire pour l'utilisateur, et un utilisateur ne peut utiliser qu'une langue à la fois; mais plusieurs utilisateurs peuvent utiliser la même langue.
\end{itemize}
\\\\
\end{tabu}

\begin{tabu}{X}
\textbf{Liées à Commande}\\
\toprule

\begin{itemize}
    \item porte le numéro: la commande étant identifiée par son numéro, cette relation est obligatoire pour la commande, la commande porte un seul numéro, et un numéro est porté par une seule commande.
    \item est commandée par: une commande est soit liée à l'utilisateur qui l'a passée, soit indépendante d'un utilisateur, dans le cas où le client n'a pas l'application et a commandé de manière conventionnelle via le serveur. Donc une commande ne peut avoir qu'un utilisateur; mais un utilisateur peut avoir plusieurs commandes au cours du temps, et une commande n'a pas obligatoirement un utilisateur.
    \item est à la table: la table sera utilisée pour livrer les détails de la commande, donc la table est obligatoire pour la commande, et une commande ne peut se trouver qu'à une table; mais plusieurs commandes peuvent se trouver à la même table, s'il y a plusieurs clients, ou au cours du temps.
    \item a été payée à: donne les date et heure auxquelles la commande a été payée; si elle n'a pas encore été payée, aucune date/heure ne sera liée. Par conséquent, une commande ne peut être payée qu'à une date; mais plusieurs commandes peuvent être payées à la même date/heure, et une commande n'a pas obligatoirement une date/heure de paiement.
\end{itemize}
\\\\
\end{tabu}

\end{document}
