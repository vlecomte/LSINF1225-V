\documentclass[a4paper,10pt]{article}

\usepackage{../../latex/mystyle}
\usepackage[top=3cm, bottom=3cm, left=3cm, right=3cm]{geometry}

\begin{document}

\header{Choix de conception}

\section{Différences par rapport à l'énoncé}

La différence principale de notre projet par rapport à l'énoncé est que le client peut lui-même commander des boissons et demander l'addition. Il n'a pas besoin de passer par l'intérmédiaire du serveur, bien que ce soit possible: dans ce cas les commandes ne seront pas associées à un utilisateur, mais seulement à la table.

\section{Description des entités}

\begin{tabu}{X}
\textbf{Utilisateur} \\
\toprule
Pour notre projet, nous avons décidé qu'il serait possible au client de commander lui-même des boissons. Dès lors, les utilisateurs de l'application peuvent être de trois types: client, serveur et admin. Ces trois rôles auront chacun leurs privilèges et leurs écrans. Afin d'utiliser l'application, un utilisateur doit s'identifier via un écran de login.

\textsl{Exemples: Tiffany, une cliente, et Jeff, un serveur.}

Les utilisateurs sont identifiés de manière unique par leur login. \\\\
\end{tabu}

\begin{tabu}{X}
\textbf{Commande} \\
\toprule

Dans notre conception, une commande représente l'ensemble des boissons consommées par un utilisateur entre deux paiements. Quand l'utilisateur commande une boisson, elle est ajoutée à la commande courante, et une fois que l'utilisateur paie, cette commande est fermée et les boissons suivantes, le cas échéant, seront ajouté à une nouvelle commande.

\textsl{Exemples: la commande \no42 par Tiffany pour sa pose de midi, elle a bu une Maes et a déjà payé; et la commande \no43 après son travail, elle a ajouté un Gin Tonic et n'a pas encore payé.}

Les commandes sont identifiées de manière unique par leur numéro de commande. \\\\
\end{tabu}

\begin{tabu}{X}
\textbf{Détail} \\
\toprule

Un détail représente une boisson ajoutée à une commande. Quand plusieurs boissons identiques sont commandées, nous créons un détail pour chacune d'entre elles. Nous considérons qu'il n'y a là pas de réelle redondance: en effet, les détails peuvent avoir été ajoutés ou servis à des moments différents, et certains d'entre eux peuvent avoir déjà été servis tandis que d'autres viennent seulement d'être ajoutés. De plus, la gestion des détails sous cette forme est significativement plus simple.

\textsl{Exemples: le détail de la commande \no42 ajouté à 11h45, une Maes 25cl, servie par Jeff à 11h47; et le détail de la commande \no43 ajouté à 18h01, un gin tonic, pas encore servi.}

Les détails sont identifiés de manière unique par la commande dont ils font partie et l'heure d'ajout du détail, pris ensemble. \\\\
\end{tabu}

\begin{tabu}{X}
\textbf{Consommation} \\
\toprule

Une consommation représente une boisson sur la carte. Si plusieurs boissons identiques sont commandées, elles pointent toutes vers la même consommation. Les consommations sont liées notamment à des types de consommation, des descriptions et des ingrédients pour les spécifier.
    
\textsl{Exemples: la consommation Maes 25cl, une bière qui coûte 1.80€; et la consommation Gin Tonic, un cocktail à 6.00€ composé de Gin et de Tonic.}

Les consommations sont identifiées de manière unique par leur nom de consommation. \\\\
\end{tabu}

\begin{tabu}{X}
\textbf{Type} \\
\toprule

Un type représente une catégorie de consommation. Le type peut-être affiché sur la carte sous forme d'une icône.

\textsl{Exemples: le type bière, le type cocktail.}

Les types sont identifiés de manière unique par leur nom de type. \\\\
\end{tabu}

\begin{tabu}{X}
\textbf{Description} \\
\toprule

Une description représente le texte descriptif d'un produit et son image. Des consommations qui ne diffèrent que par la taille mais qui sont le même produit vont partager la même description.

\textsl{Exemples: la description Maes, avec comme image une bouteille de Maes, qui décrit aussi bien un Maes 25cl qu'une Maes 30cl; la description Gin Tonic, qui ne décrit que le Gin Tonic.}

Les descriptions sont identifiées de manière unique par leur nom de description. \\\\
\end{tabu}

\begin{tabu}{X}
\textbf{Ingrédient} \\
\toprule

Un ingrédient représente le stock d'une matière première, par bouteille ou en vrac. Si une consommation utilise plusieurs ingrédients différents, il sera lié à ces ingrédients avec des relations indiquant aussi la quantité consommée. L'ingrédient possède des informations sur le stock actuel, le seuil et le maximum.

\textsl{Exemples: l'ingrédient Maes 25cl, utilisé une bouteille à la fois; l'ingrédient Gin, utilisé par petites quantités dans le Gin Tonic.}

Les ingrédients sont identifiés de manière unique par leur nom d'ingrédient. \\\\
\end{tabu}

\section{Précisions sur les relations}

\begin{tabu}{X}
\textbf{Liées à Utilisateur}\\
\toprule

\begin{itemize}
    \item possède le login: l'utilisateur étant identifié par son login, cette relation est obligatoire pour l'utilisateur, l'utilisateur a un seul login, et un login appartient à un seul utilisateur.
    \item possède l'e-mail: l'e-mail étant utilisé notamment pour récupérer un mot de passe oublié, cette relation est obligatoire pour l'utilisateur, l'utilisateur a un seul e-mail, et un e-mail appartient à un seul utilisateur.
    \item a le mot de passe: pour des raisons de sécurité, le mot de passe est obligatoire pour l'utilisateur, et un utilisateur ne peut avoir qu'un mot de passe; un mot de passe peut par contre être utilisé par plusieurs utilisateurs.
    \item utilise la langue: la langue indiquée sera utilisée pour afficher l'application, par conséquent la langue est obligatoire pour l'utilisateur, et un utilisateur ne peut utiliser qu'une langue à la fois; mais plusieurs utilisateurs peuvent utiliser la même langue.
    \item est de grade: un utilisateur doit avoir un des trois grades \val{"client"}, \val{"serveur"} et \val{"admin"}, donc le grade est obligatoire pour l'utilisateur, et un utilisateur ne peut avoir qu'un grade; mais plusieurs utilisateurs peuvent avoir le même grade.
\end{itemize}
\\\\
\end{tabu}

\begin{tabu}{X}
\textbf{Liées à Commande}\\
\toprule

\begin{itemize}
    \item porte le numéro: la commande étant identifiée par son numéro, cette relation est obligatoire pour la commande, la commande porte un seul numéro, et un numéro est porté par une seule commande.
    \item est commandée par: une commande est soit liée à l'utilisateur qui l'a passée, soit indépendante d'un utilisateur, dans le cas où le client n'a pas l'application et a commandé de manière conventionnelle via le serveur. Donc une commande ne peut avoir qu'un utilisateur; mais un utilisateur peut avoir plusieurs commandes au cours du temps, et une commande n'a pas obligatoirement un utilisateur.
    \item est à la table: la table sera utilisée pour livrer les détails de la commande, donc la table est obligatoire pour la commande, et une commande ne peut se trouver qu'à une table; mais plusieurs commandes peuvent se trouver à la même table, s'il y a plusieurs clients, ou au cours du temps.
    \item a été payée à: donne les date et heure auxquelles la commande a été payée; si elle n'a pas encore été payée, aucune date/heure ne sera liée. Par conséquent, une commande ne peut être payée qu'à une date; mais plusieurs commandes peuvent être payées à la même date/heure, et une commande n'a pas obligatoirement une date/heure de paiement.
\end{itemize}
\\\\
\end{tabu}

\begin{tabu}{X}
\textbf{Liées à Détail}\\
\toprule

\begin{itemize}
    \item est dans la commande: le détail doit être dans une commande pour être facturé, donc cette relation est obligatoire pour le détail et un détail n'appartient qu'à une seule commande; mais une commande peut contenir plusieurs détails, car elle n'est clôturée qu'une fois payée.
    \item est une: la consommation spécifie le produit que ce détail représente, donc cette relation est obligatoire pour le détail et un détail n'a qu'une seule consommation; mais une consommation peut avoir plusieurs détails, car elle peut-être commandée plusieurs fois.
    \item a été ajouté à: la date/heure d'ajout sert avec la consommation à identifier le détail, donc cette relation est obligatoire pour le détail et un détail n'a qu'une seule consommation; mais plusieurs détails peuvent être ajoutés au même moment, pourvu que leurs consommations ne soient pas les mêmes.\footnote{Si plusieurs consommations identiques sont commandées au même moment on peut imaginer conceptuellement qu'il y ait un retard dans la procédure d'ajout pour s'assurer que les moments d'ajout soient différents. Mais dans la pratique, nous utiliserons bien sûr l'ID des détails pour les distinguer.}
    \item a été servi à: la date/heure de livraison de désigne le moment où le serveur indique qu'il sert le détail, et est vide si le détail n'est pas encore servi. Par conséquent, un détail n'a qu'une seule date/heure de livraison; mais la relation n'est pas obligatoire pour le détail, et plusieurs détails peuvent être servis au même moment, ce qui est fréquent.
    \item a été servi par: donne le serveur qui s'est chargé de livrer le détail, ou rien si le détail n'est pas encore livré; donc un détail ne peut avoir qu'un seul serveur; mais la relation n'est pas obligatoire pour le détail, et plusieurs détails peuvent être servis par le même serveur, ce qui est fréquent.
\end{itemize}
\\\\
\end{tabu}

\begin{tabu}{X}
\textbf{Liées à Consommation}\\
\toprule

\begin{itemize}
    \item a pour nom: la consommation étant identifiée par son nom, cette relation est obligatoire pour la consommation, une consommation n'a qu'un seul nom, et un nom ne correspond qu'à une consommation.
    \item coûte: le prix sera utiliser pour calculer la facture d'une commande, donc cette relation est obligatoire pour la consommation, et à une consommation ne correspond qu'un prix; mais plusieurs consommations peuvent avoir le même prix.
    \item est de type: le type permettra notamment d'afficher la miniature du type dans la carte, et de trier les boissons par catégorie; donc cette relation est obligatoire pour la consommation, et à une consommation ne correspond qu'un type; mais plusieurs consommations peuvent évidemment avoir le même type.
    \item est décrite par: la description sera utilisée pour l'écran de détails / d'ajout de consommation, donc cette relation est obligatoire pour la consommation, et à une consommation ne correspond qu'une seule description; mais plusieurs consommations peuvent avoir la même description, si il s'agit du même produit à plusieurs quantités par exemple.
    \item utilise $X$ de: cette relation sera utilisée pour indiquer que la consommation fait diminuer le stock d'un produit d'une certaine quantité; c'est une relation ternaire. Il est évident qu'elle n'est ni obligatoire ni unique pour la quantité $X$ utilisée. Elle est unique pour un couple consommation-ingrédient particulier, car s'il y avait deux utilisations pour le même couple, elles pourraient être fusionnées, en utilisant pour quantité la somme des quantités. Par contre, elle n'est pas obligatoire pour la consommation, car on peut imaginer un produit, comme l'eau du robinet, qui ne dépende pas d'un stock; et elle n'est pas obligatoire pour l'ingrédient, car on peut imaginer un ingrédient disponible, mais pas ou pas encore disponible sur la carte.
\end{itemize}
\\\\
\end{tabu}

\begin{tabu}{X}
\textbf{Liées à Type}\\
\toprule

\begin{itemize}
    \item a pour nom: le type étant identifié par son nom, cette relation est obligatoire pour le type, un type n'a qu'un seul nom, et un nom ne correspond qu'à un type.
    \item est illustré par: l'icône de type sera affichée dans la carte pour identifier le type, donc cette relation est obligatoire pour le type, un type n'a qu'une seule icône, et une icône ne correspond qu'à un type.
\end{itemize}
\\\\
\end{tabu}

\begin{tabu}{X}
\textbf{Liées à Description}\\
\toprule

\begin{itemize}
    \item a pour nom: la description étant identifié par son nom ou titre, cette relation est obligatoire pour la description, une description n'a qu'un seul nom, et un nom ne correspond qu'à une description.
    \item est illustrée par: l'image de la description accompagnera le texte de la description, en est représentative, et dans notre modèle il s'agit du nom du fichier contenant l'image. Cette relation est obligatoire pour le description, une description n'a qu'une seule image, et une image ne correspond qu'à une description.
    \item est décrite par: cette relation lie la description à son texte, donc cette relation est obligatoire pour la description, une description n'a qu'un seul texte, et un texte ne correspond qu'à une description.
\end{itemize}
\\\\
\end{tabu}

\begin{tabu}{X}
\textbf{Liées à Ingrédient}\\
\toprule

\begin{itemize}
    \item a pour nom: l'ingrédient étant identifié par son nom, cette relation est obligatoire pour l'ingrédient, un ingrédient n'a qu'un seul nom, et un nom ne correspond qu'à un seul ingrédient.
    \item est d'unités: les unités de l'ingrédient permettent de déterminer ce que signifie en pratique les quantités utilisées par les consommations. Par conséquent, cette relation est obligatorie pour l'ingrédient, et un ingrédient a des unités uniques; mais plusieurs ingrédients peuvent avoir les mêmes unités.
    \item est en quantité: cette relation désigne la quantité de l'ingrédient actuellement en stock, donc cette relation est obligatoire pour l'ingrédient, et un ingrédient a une seule quantité actuelle; mais plusieurs ingrédients peuvent être actuellement aux mêmes quantités.
    \item est en quantité max.: cette relation désigne la quantité maximale de l'ingrédient que le stock peut contenir, ce qui peut ne pas être spécifié; donc un ingrédient a une seule quantité maximale, mais cette relation n'est pas obligatoire pour l'ingrédient, et plusieurs ingrédients peuvent avoir les mêmes quantités maximales.
    \item est en quantité seuil: cette relation désigne la quantité en-dessous de laquelle l'ingrédient est en quantité critique, et un avertissement est affiché aux administrateurs, mais elle est facultative. Par conséquent, un ingrédient a une seule quantité seuil, mais cette relation n'est pas obligatoire pour l'ingrédient, et plusieurs ingrédients peuvent avoir les mêmes quantités seuil.
\end{itemize}
\\\\
\end{tabu}

\section{Démarche d'écriture des faits élémentaires}

Pour l'écriture des faits élémentaires, nous avons décidé d'être le plus cohérent possible avec le reste des documents tout en respectant les contraintes des faits élémentaires.

Dès lors:
\begin{itemize}
    \item nous avons distingué entité et valeur, les premières étant écrites avec la notation abrégée et les deuxièmes en police monospace;
    \item nous avons mis les faits sous une forme structurée et régulière avec premier membre, relation et deuxième membre.
\end{itemize}

Mais:
\begin{itemize}
    \item outre les formes abrégées des entités, les phrases sont en français correct;
    \item les relations absentes sont mises en évidence dans le tableau mais ne sont pas considérées comme des faits: on n'utilise pas la valeur \val{NULL} pour les exprimer.
\end{itemize}

Grâce à ces mesures, la traduction en schéma ORM a été presque automatique, tout en gardant des faits valides.

\section{Éléments sujets à modification}

\begin{itemize}
    \item La façon dont les dates et heures sont indiquées dans les faits élémentaires, les populations représentatives et la base de donnée est une manière intuitive qui sera remplacée par des timestamps UNIX lorsque nous passerons à l'implémentation.
    \item Les noms des entités et des colonnes dans la base de donnée seront modifiés à l'implémentation, principalement à cause du passage à l'anglais.
\end{itemize}

\end{document}
