\documentclass[a4paper,10pt]{article}

\usepackage{../../latex/mystyle}
\usepackage[top=3cm, bottom=3cm, left=3cm, right=3cm]{geometry}

\begin{document}

\header{User stories}

Les durées estimées lors du \emph{planning poker} sont en unités relatives.
Toutefois, nous estimons a priori qu'une unité devrait prendre environ une demi-heure.

\section{Gestion des utilisateurs}

\story{5}{non-inscrit}
{me créer un compte utilisateur en donnant mon login, mon adresse mail et mon mot de passe}
{rentrer dans l'application et pouvoir commander des boissons}
\storyq{3}{utilisateur}
{me connecter en rentrant mon login et mon mot de passe}
{rentrer dans l'application avec mes permissions et mes données}
\story{3}{manager}
{changer le grade d'un utilisateur}
{donner les droits à un nouveau serveur ou un nouveau manager, ou diminuer les droits d'un employé qui part}
\storyq{3}{utilisateur}
{changer l'adresse mail de mon compte en donnant mon mot de passe}
{lier mon compte à la bonne adresse si je change mon adresse mail}
\storyqd{5}{utilisateur}
{changer mon mot de passe en donnant l'ancien et le nouveau deux fois}
{assurer la sécurité de mes comptes en ligne avec des changements fréquents}

\section{Affichage de la carte}

Notons que ces fonctionnalités sont aussi accessibles au serveur lorsqu'il passe commande pour une table.

\story{5}{client}
{regarder la carte des boissons}
{choisir ce que je veux boire}
\storyd{3}{client}
{savoir visuellement sur la carte quand une consommation est écoulée}
{en commander une autre}
\storyd{2}{client}
{afficher une image et la description d'une consommation}
{obtenir plus d'informations sur le produit et m'aider dans mon choix}

\section{Gestion du panier}

Chaque utilisateur a un panier auquel il peut ajouter ou enlever des boissons.
Un client peut l'ajouter à sa commande tandis qu'un serveur peut aussi choisir de l'ajouter à la commande d'une table.

\story{1}{client ou serveur}
{ajouter des consommations à mon panier}
{me souvenir de mes choix ou ceux d'une table et les confirmer plus tard}
\story{2}{client ou serveur}
{retirer des consommations de mon panier}
{corriger des erreurs de sélection ou prendre en compte des changements d'avis}
\story{3}{client ou serveur}
{afficher mon panier}
{vérifier que tout est correct}

\section{Passage de commande}

\story{3}{client}
{valider mon panier}
{passer commande auprès des serveurs et me faire livrer}
\story{3}{client}
{sélectionner ma table quand j'ajoute les premières consommations à ma commande}
{recevoir les boissons au bon endroit}
\story{3}{serveur}
{ajouter mon panier à la commande d'une table}
{gérer les commandes des clients qui n'ont pas l'application}
\story{5}{client ou serveur}
{avoir un avertissement s'il n'y a pas assez de stock pour servir tout le panier}
{savoir quels ingrédients sont en stock trop faible et modifier les choix en conséquence}

\section{Gestion du service}

\story{5}{serveur}
{voir le nombre de détails non encore servis groupés par table}
{choisir la prochaine table que je vais servir}
\story{3}{serveur}
{voir la liste des détails non encore servis pour une table}
{savoir quels consommations je dois préparer et livrer}
\storyd{2}{serveur}
{marquer les détails d'une table comme servis}
{éviter qu'un autre serveur ne les voie et ne les serve une deuxième fois}

\section{Addition et paiement}

\story{3}{client}
{afficher mon addition avec ses détails}
{savoir combien je dois payer}
\story{5}{serveur}
{afficher la liste des totaux à payer pour les clients ou les tables}
{savoir combien ils doivent payer}
\story{3}{serveur}
{afficher l'addition d'un client ou d'une table avec ses détails}
{pouvoir expliquer le prix à payer}
\storyd{1}{serveur}
{marquer une commande comme payée}
{enregistrer le paiement du client et clôturer la commande}

\section{Gestion des stocks}

\story{5}{manager}
{voir les stocks des ingrédients et leurs maxima, en mettant en évidence ceux qui sont en-dessous du seuil ou épuisés}
{déterminer quels ingrédients je dois racheter, et en quelles quantités}
\storyd{3}{manager}
{modifier les stocks enregistrés}
{actualiser les disponibilités des consommations quand je refais le plein d'un ingrédient}
\storyd{3}{manager}
{modifier les seuils et les maxima des ingrédients}
{adapter l'affichage du stock si la vitesse d'écoulement d'ingrédients ou l'espace de stockage changent}

\section{Graphiques}

La mise en place d'un système d'affichage de graphiques sera probablement un \textbf{7}.
Les durées indiquées ci-dessous représentent le temps nécessaire pour ajouter un graphique en plus.

\story{5}{manager}
{voir un graphique de l'évolution du chiffre d'affaires}
{me rendre compte de l'évolution de la popularité du bar}
\storyd{5}{manager}
{voir un graphique des boissons les plus consommées}
{acheter en plus grande quantité les ingrédients qui sont consommés le plus vite}
\story{5}{manager}
{voir un graphique de l'activité de mes meilleurs clients}
{les récompenser et les fidéliser}
\story{5}{manager}
{voir un graphique décrivant l'efficacité de mes serveurs}
{vérifier qu'ils travaillent bien, punir les tire-au-flanc et récompenser les meilleurs}

\end{document}
