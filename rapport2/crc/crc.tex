\documentclass[a4paper,10pt]{article}

\usepackage{../../latex/mystyle}
\usepackage[top=3cm, bottom=3cm, left=3cm, right=3cm]{geometry}

\begin{document}

\header{Cartes CRC}

Ces cartes CRC sont une version préliminaire de notre conception orientée objet de l'application.
Elles ne représentent pas fidèlement la structure finale du programme.

\vspace{.5cm}

\crc{Utilisateur}
{Représente tout utilisateur de l'application, c'est-à-dire un client, un serveur ou un manager.
Il correspond à une seule personne.}
{
    --- Se connecter & \\
    --- Changer ses informations & \\
}

\vspace{.7cm}

\crc{Client}
{Représente un client, c'est-à-dire quelqu'un qui peut commander des boissons dans le bar.
Les serveurs et managers peuvent aussi être des clients. Hérite de Utilisateur.}
{
    --- Afficher la carte & Carte \\
    --- Ajouter une consommation au panier & Consommation \\
    --- Retirer une consommation du panier & Consommation \\
    --- Afficher le panier & Consommation \\
    --- Confirmer le panier (créer la commande si nécessaire)
        & Commande, Détail \\
    --- Demander l'addition & Commande \\
}

\vspace{.7cm}

\crc{Serveur}
{Représente un serveur, c'est-à-dire quelqu'un qui peut servir des boissons dans le bar.
Les manageurs peuvent aussi être des serveurs. Hérite de Client.}
{
    --- Confirmer le panier pour une table (créer la commande si nécessaire)
        & Commande, Détail \\
    --- Regarder les détails commandés, groupés par table & Commande, Détail \\
    --- Voir les détails à servir pour une table & Commande, Détail \\
    --- Marquer des détails comme servis & Détail \\
    --- Regarder l'addition d'un client & Client, Commande \\
    --- Marquer une commande comme payée & Commande \\
}

\vspace{.7cm}

\crc{Manager}
{Représente un manager du bar, c'est-à-dire le propriétaire, qui a accès à toutes les fonctionnalités et qui a tous les droits.
Hérite de Serveur.}
{
    --- Changer le grade d'un utilisateur & Utilisateur \\
    --- Regarder le stock et les seuils & Stock \\
    --- Changer le stock et les seuils & Stock \\
    --- Regarder des graphiques & Graphiques \\
}

\vspace{.7cm}

\crc{Commande}
{Regroupe les consommations commandées par un utilisateur entre deux paiements.
Une commande correspond donc à une facture.}
{
    --- Ajouter les consommations venant d'un panier & Client, Consommation, Détail \\
    --- Calculer l'addition & Détail, Consommation \\
}

\vspace{.7cm}

\crc{Détail}
{Représente un élément d'une commande, c'est à dire une consommation commandée à un certain moment
et éventuellement servie à un certain moment par un serveur.}
{
    --- Se décompter du stock & Consommation, Stock \\
    --- Se marquer comme servi & Serveur \\
}

\vspace{.7cm}

\crc{Consommation}
{Représente un produit qui se trouve sur la carte, avec son prix, son type, sa description, etc.}
{
    --- Afficher sa description & Carte \\
}

\vspace{.7cm}

\crc{Ingrédient}
{Représente un ingrédient brut en stock, qui est utilisé, éventuellement en combinaison avec d'autres, pour créer des consommations.}
{
    --- Vérifier si sa quantité est en-dessous du seuil. & \\
    --- Se refournir d'une certaine quantité & \\
}

\vspace{.7cm}

\crc{Carte}
{Représente l'ensemble des consommations disponibles dans le bar.}
{
    --- Afficher la liste des consommations & Consommation \\
    --- Trier les boissons par type & Consommation \\
    --- Se rafraîchir en fonction des stocks & Consommation, Stock \\
    --- Afficher une description & Consommation \\
}

\vspace{.7cm}

\crc{Stock}
{Contient les différents ingrédients qui représentent l'état du stock.}
{
    --- Afficher le stock & Manager \\
    --- Vérifier si un panier prend trop d'ingrédients, et retourner les ingrédients en quantité insuffisante & Consommation, Ingrédient \\
    --- Décompter un panier du stock & Consommation, Ingrédient \\
}

\vspace{.7cm}

\crc{Graphiques}
{Compile des données et affiche des graphiques}
{
    --- Graphique de l'évolution du chiffre d'affaires & Commande \\
    --- Graphique des boissons les plus consommées & Consommation, Détail \\
    --- Graphique des meilleurs clients & Client, Commande \\
    --- Graphique de l'efficacité des serveurs & Serveur, Détail \\
}

\end{document}
